
\documentclass[10pt,a4paper]{article}

\usepackage[polish]{babel}
\usepackage{polski}
\usepackage[utf8]{inputenc}
\usepackage[T1]{fontenc}
\usepackage{graphicx}
\frenchspacing

\title{Inżynieria oprogramowania\\zwierzaki.pl}
\author{Marcin Nowak\\Paweł Obrok\\Paweł Pierzchała}
\date{Kompilacja dokumentu: \today}
 
\begin{document}
\maketitle
\clearpage

\tableofcontents
\clearpage

\section{Wizja}

Celem portalu będzie zrzeszanie oraz usprawnienie akcji podejmowanych przez organizacje oraz wolantariuszy pomagających zwierzętom z całej Polski.

\subsection{Zwierzęta}

Głównym zadaniem portalu będzie katalogowanie bezdomnych zwierząt, pośrednictwo w ich adopacjach, zarówn wirtualnych jak i rzeczywistych oraz śledzenie ich losów. Na stronie każdego zwierzęcia będą się znajdowały podstawowe informacje, tzn.:
\begin{enumerate}
	\item dane (rasa, imię, namaszczenie)
	\item obszar, w którym zwierzę aktualnie przebywa
	\item informacje o wynikach ewentualnych badań weterynaryjnych i/lub chorobach
	\item zdjęcia
\end{enumerate}

Poza tym każde zwierzę będzie posiadało swój "kalendarz zdarzeń". Będzie on mógł zawierać m.in.:
\begin{enumerate}
	\item orientacyjne daty narodzin/odnalezienia
	\item daty wizyt u weterynarza
	\item datę adopcji i rodzaj adopcji
	\item datę ewentualnej zmiany miejsca zamieszkania (np. gdy zwierzę przeprowadzi się razem ze swoim nowym panem)
	\item ewentualną datę śmierci
\end{enumerate}

\subsection{Adopcje}

Rodzaje adopcji:
\begin{enumerate}
	\item adopcja wirtualna polega na zapewnieniu środków finansowych na utrzymanie zwierzęcia, np. operacje na które schronisko nie ma funduszy
	\item rzeczywiste polegające na zabraniu zwierzęciu do domu
\end{enumerate}

\subsection{Użytkownicy}

\subsubsection{Wolontariusze}
Zdecydowaną większość użytkowników portalu będą stanowili wolontariusze. Będą oni mogli dodawać informacje zwierzętach (zarówno odnalezionych jak i znajdujących się w bazie), zdjęcia oraz wydarzenia w kalendarzu. Wszyscy wolonatariusze portalu będą mogli dokonowayć adopcji. Wiele osób opiekujących się zwierzętami nie korzysta z internetu, dlatego wolontariusze powinni mieć możliwość pośrednicznia w takich adopcjach.
 
\subsubsection{Organizacje}
Swoje profile na zwierzaki.pl będą miały organizacje pomagające bezdomnym zwierzętom (np. schroniska). Na stronach organizacji będą znajdowały się m.in.:
\begin{enumerate}
	\item profil działalności wraz z opisem
	\item lokalizacja
	\item historia przeprowadzanych adopcji
	\item informacje o nadchodzących wydarzeniach (np. zbiórka jedzenia czy zabawek dla schroniska, spotkanie z dyskusją na temat zwierząt, spotkania w szkołach)
\end{enumerate}

\subsubsection{Weterynarze}
Specjalne profile będą mogli zakładać weterynarze. Na swoich stronach będą zamieszczać 
informacje o przychodni oraz zaznaczać, którymi zwierzętami się opiekują. Tylko weterynarze 
i upoważnieni przez nich wolontariusze będą mogli edytować status choroby zwierzęcia oraz 
dodawać informacje o leczeniu do kalendarza.

\subsection{Mapy}
Ze względu na to, że aktywność użytkowników jest ściśle związane z obszarem, w którym się oni znajdują, portal będzie zawierał mapy. Na mapach będą zaznaczone "punkty przyjazne dla zwierzaków" czyli m.in. schroniska, przychodnie weterynaryjne, siedziby organizacji. Poza tym będzie można zaznaczać orientacyjne obszary, w których znajdują się określone zwierzęta.

\end{document}
